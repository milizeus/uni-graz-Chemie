%!TEX root = ../chemie.tex



\newpage

\chapter{CHEMISCHE KINETIK}
\begin{enumerate}

	%%%%%%%%%%%%%%%%%%%%%%%%%%%%%%%%%%%%%%%%%%%%%%%%%%%%%%%%
	
	\item F�r die Reaktion \ce{BF_{3(g)} + NH_{3(g)} -> F_{3}BNH_{3(g)}}
	wurden folgende Daten gemessen:\\
	 %
	\begin{tabular}{cccc}
	\hline 
	Versuch & \ce{BF_{3}} / $\frac{M}{L}$ & \ce{NH3} / $\frac{M}{L}$ & Anfangsreaktionsgeschw $\frac{M}{s}$ \tabularnewline
	\hline 
	\hline 
	1 & 0,25 & 0,25 & 0,1230\tabularnewline
	\hline 
	2 & 0,250 & 0,125 & 0,1065\tabularnewline
	\hline 
	3 & 0,200 & 0,100 & 0,0682\tabularnewline
	\hline 
	4 & 0,350 & 0,100 & 0,1193\tabularnewline
	\hline 
	5 & 0,175 & 0,100 & 0,0596\tabularnewline
	\hline 
	\end{tabular}\\
	Wie lautet das Geschwindigkeitsgestz f�r die Reaktion? Was ist die
	Gesamtordnung der Reaktion? Was ist der Wert f�r die Geschwindigkeitskonstante
	der Reaktion?
	
	\begin{enumerate}
	\item a
	\end{enumerate}
	\item Die Aktivierungsenergie einer bestimmten Reaktion ist $65.7kJ/Mol$.
	Wie viel schneller findet die Reaktion bei Die Zersetzung von Wasserstoffperoxid
	wird durch Jodidionen katalysiert. Man glaubt, dass die katalysierte
	Reaktion �ber einen zweistufigen Mechanismus abl�uft:
	\ce{H2O2 + I- -> H2O + IO-} (langsam) und \ce{IO- + H2O2 -> H2O + O2 + I-}
	(schnell) Schreiben Sie das Geschwindigkeitsgesetz f�r jede der
	Elementarreaktionen des Reaktionsmechanismuses an. 
	Schreiben Sie die chemische Gleichung f�r den Gesamtprozess. 
	Sagen Sie das Geschwindigkeitsgesetz f�r den Gesamtprozess vorher.
	
	\begin{enumerate}
	\item a
	\end{enumerate}
	\item Der erste Schritt eines Mechanismus bei der Reaktion von Brom ist:
	\ce{Br2 <-> 2Br} (schnell, im Gleichgewicht)\\
	Wie lautet der Ausdruck, der die Konzentration von \ce{Br} mit der von
	\ce{Br2} in Beziehung setzt?
	
	\begin{enumerate}
	\item a
	\end{enumerate}
	\item Viele metallische Katalysatoren, vor allem Edelmetallkatalysatoren,
	werden h�ufig als sehr d�nne Schichten auf einer Substanz mit hoher
	spezifischer Oberfl�che, wie Aluminiumoxid oder Siliziumoxid abgeschieden.
	Warum ist das ein effektives Verfahren zur Nutzung von Katalysatorstoffen?
	
	\begin{enumerate}
	\item a
	\end{enumerate}
	\end{enumerate}
	
	

%%%%%%%%%%%%%%%%%%%%%%%%%%%%%%%%%%%%%%%%%%%%%%%%%%%%%%%%




