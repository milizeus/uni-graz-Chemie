%!TEX root = ../chemie.tex



%\paragraph{LIMITIERENDE REAKTANTEN - ST�CHIOMETRIE}
\chapter{LIMITIERENDE REAKTANTEN - ST�CHIOMETRIE}
\begin{enumerate}

	%%%%%%%%%%%%%%%%%%%%%%%%%%%%%%%%%%%%%%%%%%%%%%%%%%%%%%%%

	\item 
	Stellen Sie sich vor, Sie wollen einen Prozess verbessern, in dem
	aus Eisenerz, das \ce{Fe_{2}O_{3}} enth�lt, Eisen gewonnen wird. Als
	Testexperiment f�hren sie folgende Reaktion durch: \ce{Fe_{2}O_{3} +3CO -> 2Fe +3CO_{2}}\\
	Schreiben Sie f�r alle Atome die Oxidationszahlen an. \\
	Wenn Sie mit 150 g \ce{Fe_{2}O_{3}} limitierendem Reaktanten beginnen, wie gro� ist
	die theoretische ausbeute an \ce{Fe}? \\
	Wie gro� w�re die prozentuale Ausbeute, wenn Ihre tats�chliche Ausbeute an \ce{Fe} 87.9 g w�re?
	
	\begin{enumerate}
	
		\item
		Oxidationszahlen: 
		\ce{Fe2^{3+} O3 ^{2-} +3C ^{2+} O ^{2-} -> 2Fe^{0} + 3C^{4+}O2^{2-}}
		
		\item
		theoretische Ausbeute:\\ \\
		Mol berechnen, Stoffmenge mit Atomgewicht multiplizieren\\
		theoretische Ausbeute= $x  Mol \cdot x \frac{g}{Mol}$\\ 
		
		$M_{\ce{Fe2O3}}= 2\cdot55,8+3\cdot16= 159,6 g/Mol$ \\
		$M_{\ce{3CO}}=3\cdot(12+16)=84 g/Mol$\\
		$M_{\ce{2Fe}}=2\cdot55,8=111,6g/Mol$\\
		$M_{\ce{3CO2}}=3(12+2\cdot16)=132 g/Mol$\\
		
		Mol bei 150g \ce{Fe2O3}: $x=\frac{150g}{159,6\frac{g}{Mol}}=0,94Mol$\\
		\ce{Fe2O3 -> 2Fe} \\
		$0,94Mol\cdot111,6 \frac{g}{Mol}=104,9g$
		
		\item
		prozentuale Ausbeute:\\
		$\frac{tatsaechliche Ausbeute}{theoretische Ausbeute} \cdot 100=\frac{87,9g}{104,9g}\cdot100=83,79\%$
		
	\end{enumerate}

	%%%%%%%%%%%%%%%%%%%%%%%%%%%%%%%%%%%%%%%%%%%%%%%%%%%%%%%%
	
	\item
	Fluorwasserstoffs�ure kann nicht in Glasbeh�ltern aufbewahrt werden,
	weil Silikate von \ce{HF} angegriffen werden. Natriumsilikat reagiert
	z.B. wie folgt: \\
	\ce{Na_{2}SiO_{3} +8HF -> H_{2}SiF_{6} +2NaF +3H_{2}O} \\
	Wie viel Mol \ce{HF} werden ben�tigt, um mit 0.300 Mol \ce{Na_{2}SiO_{3}}
	zu reagieren? \\
	Wieviel g \ce{NaF} werden gebildet, wenn 10 g \ce{HF} mit
	85.4 g \ce{Na_{2}SiO_{3}} reagieren? 
	
	\begin{enumerate}
	
		\item 
		$n_{\ce{HF}}=0,3\cdot8=2,4Mol$
		
		\item
		$M_{\ce{Na_{2}SiO_{3}}}=(2\cdot23)+28+(3\cdot16)=122 \frac{g}{Mol}$\\
		$n_{\ce{Na_{2}SiO_{3}}}=\frac{85,4g}{122 \frac{g}{Mol}}=0,7Mol$ \\ 
		
		$M_{\ce{HF}}=1+19=20\frac{g}{Mol}$\\
		$n_{\ce{HF}}=\frac{10g}{20\frac{g}{Mol}}=0,5Mol < 0,7Mol \Rightarrow begrenzenter Teil$\\ 
		
		$M_{\ce{NaF}}=23+19=42\frac{g}{Mol}$\\ \\
		$\frac{2\cdot42}{8\cdot20}=\frac{m_{\ce{NaF}}}{10g} \Rightarrow m_{\ce{NaF}}=5,25g$
		
	\end{enumerate}
	
	%%%%%%%%%%%%%%%%%%%%%%%%%%%%%%%%%%%%%%%%%%%%%%%%%%%%%%%%
	
	\item
	Wenn ein Gemisch aus 10.0 g Acetylen $(H-C\text{\ensuremath{\equiv}}C-H)$
	und 10 g Sauerstoff entz�ndet wird, entsteht bei der Verbrennungsreaktion
	\ce{CO_{2}} und \ce{H_{2}O}. \\
	Bestimmen sie die Oxidationsstufen aller beteiligten Atome. \\
	Geben sie die ausgeglichene Gleichung dieser Reaktion an. \\
	Welcher Reaktant begrenzt die Reaktion? \\
	Wie viel Gramm jedes Reaktionspartners sind nach der Reaktion vorhanden? 
	
	\begin{enumerate}
	
		\item
		\ce{2H2^{1+}C2^{1-} + 5O2 -> 4C^{4+}O2^{2-} + 2H2^{1+}O^{2-} }
		
		\item
		$M_{\ce{H2C2}}=(2\cdot1)+2\cdot12=26 \frac{g}{Mol}$\\
		$n_{\ce{H2C2}}=\frac{10g}{26 \frac{g}{Mol}}=0,385Mol$ \\ 

		$M_{\ce{O2}}=(2\cdot16)=32 \frac{g}{Mol}$\\
		$n_{\ce{O2}}=\frac{10g}{32 \frac{g}{Mol}}=0,312Mol\Rightarrow \textnormal{begrenzenter Teil}$ \\ 
		
		$M_{\ce{CO2}}=(12+2\cdot16)=44 \frac{g}{Mol}$\\
		$M_{\ce{H2O}}=(2\cdot1+16)=18 \frac{g}{Mol}$\\
		
		\item
		$m_{\ce{O2}}		=0g \Leftarrow \textnormal{wird komplett aufgebraucht}$ \\
		$m_{\ce{H2C2}}	=10g-\frac{2\cdot0,312}{5}\cdot26		=6,76g$ \\
		$m_{\ce{CO2}}		=\frac{4\cdot0,312}{5}\cdot44		=10,98g$\\
		$m_{\ce{H2O}}		=\frac{2\cdot0,312}{5}\cdot18		=2,25g$\\
		
	\end{enumerate}

	%%%%%%%%%%%%%%%%%%%%%%%%%%%%%%%%%%%%%%%%%%%%%%%%%%%%%%%%

	\item 
	Eine Probe von 70.5 mg \ce{K3PO4} wird zu 15.0 mL 0.4 molarer \ce{AgNO3} -L�sung gegeben.
	Es f�llt ein Niederschlag aus. \\
	Wie lauten die Namen von \ce{K3PO4} und \ce{AgNO_{3}}? \\
	Geben sie die Gleichung f�r diese Reaktion an. \\
	Berechnen Sie die theoretische Ausbeute des gebildeten Niederschlags in Gramm.\\
	Schreiben Sie den allgemeinen Ausdruck f�r das L�slichkeitsprodukt des Niederschlags an.
	
	\begin{enumerate}
	
		\item 
		 \ce{K3PO4}: 		Kaliumphosphat, 
		 \ce{AgNO_{3}}:	Silbernitrat \\
		 
		 \item
		 \ce{K3^{1+}P^{5+}O4^{2-} + 3 Ag3^{1+}N^{3+}O3^{2-} -> Ag3^{1+}P^{5+}O4^{2-} + 3 K^{1+}N^{5+}O3^{2-}}
		 
		 \item
		 $M_{\ce{K3PO4}}	=(3\cdot39)+31+(4\cdot16))	=212\frac{g}{Mol}$\\
		 $M_{\ce{AgNO3}}	=108+14+(3\cdot16)	=170\frac{g}{Mol}$ \\
		 $M_{\ce{Ag3PO4}}	=(3\cdot108)+31+(4\cdot16)	=419\frac{g}{Mol}$ \\
		 %$M_{\ce{KNO3}}	=39+14+(3\cdot16)		=101\frac{g}{Mol}$
		 
		 $n_{\ce{K3PO4}}	= \frac{70,5\cdot10^{-3}}{212}	=0,0003325Mol$ \\
		 $0,4$ molare L�sung $\Rightarrow$ 0,4 Mol pro Liter \\
		 $n_{\ce{AgNO3}}	= 0,4\cdot15\cdot10^{-3}		=6\cdot10^{-3}Mol$\\
		 
		 $m_{\ce{Ag3PO4}}	=6\cdot10^{-3}\cdot419			=2,5g$
		 
	
	\end{enumerate}

	%%%%%%%%%%%%%%%%%%%%%%%%%%%%%%%%%%%%%%%%%%%%%%%%%%%%%%%%
	
	\item 
	Eine Probe aus festem \ce{Ca(OH)_{2}} wird bei 30�C in Wasser
	ger�hrt, bis die L�sung mit \ce{Ca(OH)_{2}} ges�ttigt ist. 100 mL dieser
	Probe werden entnommen und mit $5,00\cdot10^{-2}$ Mol/L \ce{HBr}-L�sung titriert.
	Zur Neutralisation der Probe werden 48.8 mL der S�urel�sung ben�tigt. \\
	Welche Konzentration hat die \ce{Ca(OH)_{2}}-L�sung? \\
	Wie gro� ist bei 30�C die L�slichkeit von \ce{Ca(OH)_{2}} in Wasser (Angabe
	in g \ce{Ca(OH)_{2}} pro Liter).
	
	\begin{enumerate}

		\item
		\ce{Ca(OH)2 + 2HBr -> CaBr2 + 2H2O}\\
		$n_{\ce{HBr}}		=5,00\cdot10^{-2} \frac{Mol}{L}\cdot48,8\cdot10^{-3}L=0,00244Mol$ \\
		$n_{\ce{Ca(OH)2}}	=\frac{n_{\ce{HBr}}}{2}	=\frac{0,00244Mol}{2}	=0,00122Mol$\\
		$\ce{Ca(OH)2}	=\frac{0,00122Mol}{100\cdot10^{-3}L}	=0,0122\frac{Mol}{L}$
		
		\item
		$M_{\ce{Ca(OH)2}}	=40+2\cdot(16+1)	=74\frac{g}{Mol}$\\
		$\ce{Ca(OH)2}	=0,0122\frac{Mol}{L} \cdot 74\frac{g}{Mol}		=0,903\frac{g}{L}$

	\end{enumerate}

	%%%%%%%%%%%%%%%%%%%%%%%%%%%%%%%%%%%%%%%%%%%%%%%%%%%%%%%%

\end{enumerate}


