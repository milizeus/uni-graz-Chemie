%!TEX root = ../chemie.tex



\newpage

\chapter{PERIODISCHE EIGENSCHAFTEN \textendash{} CHEMISCHE BINDUNG - BINDUNGSTTHEORIE}
\begin{enumerate}
	
	%%%%%%%%%%%%%%%%%%%%%%%%%%%%%%%%%%%%%%%%%%%%%%%%%%%%%%%%
	
	\item 
	Wie ist die generelle Beziehung zwischen der Gr��e eines Atoms und
	seiner ersten Ionisierungsenergie? Welches Element hat die gr��te
	Ionisierungsenergie? Welches die kleinste?
	
	\begin{enumerate}

		\item 
		Je gr�{\ss}er das Atom desto kleiner seine Ionisierungsenergie. \\
		Ionisierungsenergie ist die Energie, die ben�tigt wird, um ein Atom oder Molek�l zu ionisieren, 
		d. h. um ein Elektron vom Atom oder Molek�l zu trennen. 
		Allgemein ist die n-te Ionisierungsenergie die Energie, die ben�tigt wird, um das n-te Elektron zu entfernen. \\
		d.h.: Die Ionisierungsenergie nimmt im Allgemeinen von links nach rechts zu und nimmt von oben nach unten ab. \\
		kleinste: \ce{He} \\
		gr�{\ss}te: \ce{Fr}

	\end{enumerate}
	
	%%%%%%%%%%%%%%%%%%%%%%%%%%%%%%%%%%%%%%%%%%%%%%%%%%%%%%%%
	
	\item 
	Identifizieren Sie das Element, dessen Ionen die folgende Elektronenkonfiguration
	haben: \\
	a) ein 2+Ion mit $[\ce{Ar}]3d^{9}$, \\
	b) ein 1+Ion mit $[\ce{Xe}]4f^{14}5d^{10}6s^{2}$.\\
	c) Wie viele freie Elektronen besitzen die Ionen?
	
	\begin{enumerate}

		\item
		$[\ce{Ar}]3d^{9} + 2e^{-} 				\longrightarrow [\ce{Ar}]3d^{10}4s^{1}				=\ce{Cu}$
		
		\item
		$[\ce{Xe}]4f^{14}5d^{10}6s^{2} + e^{-} 	\longrightarrow [\ce{Xe}]4f^{14}5d^{10}6s^{2}6p^{1}		=\ce{Tl}$
		
		\item
		\textcolor{red}{hm? }
		

	\end{enumerate}
	
	%%%%%%%%%%%%%%%%%%%%%%%%%%%%%%%%%%%%%%%%%%%%%%%%%%%%%%%%
	
	\item 
	Warum ist Kalzium im Allgemeinen reaktiver als Magnesium? Warum ist
	Kalzium im Allgemeinen weniger reaktiv als Kalium?
	
	\begin{enumerate}

		\item
		Je gr�{\ss}er das Atom desto kleiner seine Ionisierungsenergie. \\
		Die Ionisierungsenergie nimmt im Allgemeinen von links nach rechts zu und nimmt von oben nach unten ab

	\end{enumerate}
	
	%%%%%%%%%%%%%%%%%%%%%%%%%%%%%%%%%%%%%%%%%%%%%%%%%%%%%%%%
	
	\item 
	Orden sie die folgenden Elemente nach ihrem Schmelzpunkt: $K,Br_{2},Mg,O_{2}$.
	Erkl�ren sie die Faktoren, die die Reihenfolge bestimmen.
	
	\begin{enumerate}

		\item
		\ce{O2}(-218�C) < \ce{Br2}(-7�C) < \ce{K}(63.3�C) < \ce{Mg}(639�C)
		
		\item
		Bei der Metallenbindung(Elektronengas) nimmt die Anziehung zwischen Elektronengas und Atomkernen 
		mit der Anzahl der Au�enelektronen zu also \ce{Mg} > \ce{K}
		
		\item
		

	\end{enumerate}
	
	%%%%%%%%%%%%%%%%%%%%%%%%%%%%%%%%%%%%%%%%%%%%%%%%%%%%%%%%
	
	\item 
	Phosgen hat folgende Elementzusammensetzung: 12,14\% \ce{C}, 16,17\% \ce{O} und 71,69\% \ce{Cl} (Massenprozent) 
	und ein Molekulargewicht von $98.9\: g/Mol$.\\
	Bestimmen sie die Molek�lformel dieser Verbindung. \\
	Zeichnen sie die Lewis- Strukturformel dieser Verbindung und argumentieren Sie warum
	Sie eine Bestimmte (von mehreren m�glichen) ausgew�hlt haben.
	
	\begin{enumerate}

		\item
		$M_{\ce{C}}	=12\frac{g}{Mol}$\\
		$M_{\ce{O}}	=16\frac{g}{Mol}$\\
		$M_{\ce{Cl}}	=35,5\frac{g}{Mol}$\\
		$\Rightarrow \ce{COCl2}$
		
		\item
		Man w�hlt die Lewis-Strukturformel, in der die Formalladungen der Atome am N�chsten bei Null sind.\\
		Man w�hlt die Lewis Strukturformel, in der sich die negativen Ladungen an den elektronegativeren Atomen befinden.\\
		\includegraphics[width=0.15\textwidth]{_images/Phosgen.png}


	\end{enumerate}
	
	%%%%%%%%%%%%%%%%%%%%%%%%%%%%%%%%%%%%%%%%%%%%%%%%%%%%%%%%
	
	\item Ein unbekannter Stoff liefert eine Elementaranalyse von: \ce{C}: 68.13\%,
	\ce{H}: 13.72\%, \ce{O}: 18.15\% (Massenprozent). \\
	Bestimmen sie die Molek�lformel dieser Verbindung. \\
	Zeichnen sie mindestens drei reale Molek�le, die der Molek�lformel entsprechen.
	
	\begin{enumerate}

		\item 
		$M_{\ce{C}}	=12\frac{g}{Mol}$\\
		$M_{\ce{H}}	=1\frac{g}{Mol}$\\
		$M_{\ce{O}}	=16\frac{g}{Mol}$\\ 
		
		$\ce{C}:	\frac{68,13}{12}		=5,72Mol\%$\\
		$\ce{H}:	\frac{13,72}{1}		=13,72Mol\%$\\
		$\ce{O}:	\frac{18,15}{16}		=1,13Mol\%$\\ 
		
		$5,72 : 13,72 : 1,13 \Rightarrow 5:12:1 \Rightarrow \ce{C5H12O}$
		
		\item

	\begin{minipage}{0.2\textwidth}%
		\centering
		\includegraphics[width=\textwidth]{_images/Pentanol}%
	\end{minipage}%
	\hspace{1cm}
	\begin{minipage}{0.2\textwidth}%
		\centering
		\includegraphics[width=\textwidth]{_images/2-Methyl-1-butanol}
	\end{minipage}%
	\hspace{1cm}
	\begin{minipage}{0.2\textwidth}%
		\centering
		\includegraphics[width=\textwidth]{_images/3-Methyl-1-butanol}%
	\end{minipage}%
		
		
	\end{enumerate}
	
	%%%%%%%%%%%%%%%%%%%%%%%%%%%%%%%%%%%%%%%%%%%%%%%%%%%%%%%%

	\item Beschreiben Sie anhand einer Skizze s�mtliche Bindungen in Ethylen
	mit Hilfe des Konzepts der Hybridisierung. Bezeichnen Sie die Orbitale
	die Sie zeichnen.
	
	\begin{enumerate}
	
		\item 
		Eine sigma-bindung ist eine einfachbindung zwischen zwei Atomen. 
		
		hier im ethen w�ren das die bindungen zu den wasserstoffatomen 
		und eine bindung zwischen den kohlenstoffatomen. 
		
		die $\pi$-bindung ist bei doppel- und dreifachbindungen vorhanden. zus�tzlich zur einen $\sigma$-bindung. 
		
		bei der $\pi$-bindung �berlappen zwei p-orbitale, also nicht hybridisierte...  
		
		da im ethen ja nur eine doppelbindung vorhanden ist, brauchst du also nur ein nichthybridisiertes p-orbital pro 
		Kohlenstoffatom. 
		also k�nnen die zwei anderen p-orbitale mit dem s orbital hybridisieren und man erh�lt: 
		sp2 davon h�tten wir ja dann drei st�ck pro C-Atom. 
		
		und das passt ja dann auch: zwei bindungen zu wasserstoffatomen und eine bindung zum anderen kohlenstoffatom. 
		und die zwei p-orbitale �berlappen einander und bilden zweite bindung, die doppelbindung

		\includegraphics[width=0.2\textwidth]{_images/Ethen.png}	% angle width height scale
	
	\end{enumerate}
	
	%%%%%%%%%%%%%%%%%%%%%%%%%%%%%%%%%%%%%%%%%%%%%%%%%%%%%%%%
	
	\item Zeichnen sie das Molek�lorbitalschema f�r \ce{O2}.
	
	\begin{enumerate}
	\item \includegraphics[width=0.5\textwidth]{_images/sauermo.png}
	
	\end{enumerate}
	
	%%%%%%%%%%%%%%%%%%%%%%%%%%%%%%%%%%%%%%%%%%%%%%%%%%%%%%%%
	
	\item Eine Verbindung, die aus 2.1\% \ce{H}, 29.8\% \ce{N} und 68,1\% \ce{O} besteht, hat
	ein Molekulargewicht von ca. 50 g/Mol. \\
	Wie lautet die Molek�lformel der Verbindung? \\
	Zeichnen sie die Lewis-Formel, wenn \ce{H} an \ce{O} gebunden ist. \\
	Wie ist die Struktur des Molek�ls? \\
	Wie ist die Hybridisierung der Orbitale am \ce{N}-Atom? \\
	Wie viele $\sigma$ - und $\pi$ -Bindungen gibt es in dem Molek�l?
	
	\begin{enumerate}
	
		\item 
		$M_{\ce{H}}	=1\frac{g}{Mol}$\\
		$M_{\ce{N}}	=14\frac{g}{Mol}$\\
		$M_{\ce{O}}	=16\frac{g}{Mol}$\\ 
		
		$\ce{H}:	\frac{2,1}{1}		=2,1Mol\%$\\
		$\ce{N}:	\frac{29,8}{14}		=2,13Mol\%$\\
		$\ce{O}:	\frac{68,1}{16}		=4,26Mol\%$\\ 
		
		$2,1 : 2,13 : 4,26 \Rightarrow 1:1:2 \Rightarrow \ce{HNO2}$
		
		\item
		\includegraphics[width=0.2\textwidth]{_images/Lewis_Formel_Salpetrige_Saeure.png}
		
		\item
		Die Struktur ist trigonal eben.
		
		\item
		sp$^{2}$-Hybridisierung um N
		
		\item
		3 $\sigma$-Bindungen und 1 $\pi$-Bindung
		
	
	\end{enumerate}

	%%%%%%%%%%%%%%%%%%%%%%%%%%%%%%%%%%%%%%%%%%%%%%%%%%%%%%%%
	
\end{enumerate}



